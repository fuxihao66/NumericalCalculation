\documentclass{ctexart}
\usepackage{ctex}
\usepackage{listings} 
\usepackage{setspace}
\usepackage{geometry}
\geometry{left=2.5cm,right=2.0cm,top=2.5cm,bottom=2.5cm}
%标题左对齐
\CTEXsetup[format={\Large\bfseries}]{section}
\title{\textbf{数值分析 第一次大作业}}
\author{傅锡豪, 15091082}

\begin{document}
\begin{spacing}{1.8}
\maketitle

\section{算法设计}
本程序的算法包括以下四个方面\\
\textbf{一、存储: }
由于题目的矩阵是稀疏的带状矩阵,所以通过把矩阵逆时针旋转储存,大大减少存储空间。原矩阵和被压缩矩阵之间元素的对应关系是:$a_{i,j} = c_{i-j+3,j}$。原本是501x501的矩阵,经过压缩后是5x501的矩阵,而且只有$c_{1,1}\ c_{1,2}\ 
c_{2,1}\ c_{501,501}\ c_{501,500}\ c_{500,501}$是不存储数据的,其余空间都存放了原矩阵的非0数据。所以后面使用幂法和反幂法求解特征值的时候都是使用针对\textbf{带状矩阵}的算法。\\
\textbf{二、计算特征值: }
对于该题,需要分别求解最大、最小、按模最小、和距离特定39个值最近的特征值,分别由以下算法给定:\\
(1)\textbf{求解按模最大特征值:}\\
直接对矩阵A使用幂法,可直接求解出按模最大的特征值$\lambda_{max}$。\\
(2)\textbf{通过平移求出另一特征值:}\\
用(1)求的$\lambda_{max}$的绝对值对矩阵A做平移,对$(A-|\lambda_{max}|I)$进行反幂法操作得到$1/\beta$,计算$\lambda = 1/\beta +|\lambda_{max}|$。通过比较$\lambda_{max}$和$\lambda$的值的大小,把数值大的赋给$\lambda_{501}$,数值小的赋给$\lambda_{1}$。\\
(3)\textbf{求按模最小特征值}\\
对矩阵A用反幂法直接求出按模最小的特征值,即得到$\lambda_{s}$。\\
(4)\textbf{求和$\mu_k$相近的特征值}\\
进入循环:k: 1 $\rightarrow$ 39,依次执行以下操作:\\
	1. 求解$\mu_k$:$\mu_k = \lambda_1 + k*(\lambda_{501}-\lambda{1})/40$;\\
	2. 对用$\mu_k$的值矩阵A平移,对$(A-\mu_kI)$做反幂法操作,得到$1/\beta_k$;\\
	3. 距离$\mu_k$最近的特征值由下式给定: $\lambda_{\mu_k} = \mu_k + 1/\beta_k$ 
\\
\textbf{三、A的条件数}\\
由非奇异对称矩阵的条件数由该式给定:$cond(A)_2 = |\lambda_1 / \lambda_n|$,其中$\lambda_1$和$\lambda_n$分别是该矩阵的模最大和模最小的特征值。\\
在该题中,$cond(A)_2 = |\lambda_{max} / \lambda_s|$。\\
\textbf{四、A的行列式}\\
本题中A的阶数过大,无法直接求解行列式。采用将A进行Doolittle分解,则$|A| = |L||U|$。由于L是对角线为1的下三角阵,而U是三上角阵,所以$|A| = |L||U| = \Pi_{i=1}^{501} u_{i,i}$。\\
在本题中,由于使用反幂法时对A做了Doolittle分解,所以在对矩阵A用反幂法求出$\lambda_{s}$后,把被处理过的A第3行(相当于未被压缩矩阵的对角线)各元素相乘,即 $det(A) = \Pi_{i=1}^{501} c_{3,i}$。
\end{spacing}

\begin{spacing}{1.4}
\section{源程序}
\lstset{
frame=shadow
numbers=left,
stepnumber=1,
breaklines=true,
tabsize=4,
numbersep=10pt,
numbers=left, 
numberstyle=\tiny,
basicstyle=\footnotesize
} 

\begin{lstlisting}[language={[ANSI]C++}] 
#include <stdio.h>
#include <stdlib.h>
#include <iostream>
#include <float.h>
#include <iomanip>
#include <cmath>
#define MEle A->Elements
#define VEle b->Elements
#define SEle Solution->Elements
#define YEle y->Elements
#define UEle u->Elements
#define TEle tp->Elements
#define AREle A_remain->Elements
using namespace std;

int main()
{
	double Epthron = 10e-12;
	double c, b;
	double Miu[39];
	double a[501];
	double DET_A = 1;
	double lambda_s;
	double lambda_max;
	double lambda1;
	double lambda501;
	double cond;
	int s = 2;
	Matrix* A = new Matrix(5, 501); 
	//用于保护矩阵
	Matrix* A_remain = new Matrix(5, 501);
	A->setR(2);
	A->setS(2);

	for (int j = 0; j < 501; j++){
		a[j] = (1.64-0.024*(j+1))*sin(0.2*(j+1))-0.64*exp(0.1/(j+1));
	}
	b = 0.16;
	c = -0.064;
	//init the MATRIX A
	for (int j = 0; j < 501; j++){
		if (j > 1) {
			MEle[0][j]  = c;	
			AREle[0][j] = c;
		}
		if (j > 0){
			MEle[1][j]  = b;
			AREle[1][j] = b;
		}
		if (j < 500){
			MEle[3][j] 	= b;
			AREle[3][j] = b;
		}
		if (j < 499){
			MEle[4][j] 	= c;
			AREle[4][j] = c;
		}
		MEle[2][j] 	= a[j];
		AREle[2][j] = a[j];
	}

	//get the one with max module
	lambda_max = PowerMethod(A, Epthron);

	//translate to get the other
	A->Translate(abs(lambda_max));
	lambda1 = InversePowerMethod(A, Epthron) + abs(lambda_max);

	//compare lambda1 and lambda501
	if (lambda1 > lambda_max){
		lambda501 = lambda1;
		lambda1 = lambda_max;
	}
	else {
		lambda501 = lambda_max;
	}

	//get lambda_s
	Recover(A_remain, A);
	lambda_s = InversePowerMethod(A, Epthron);

	//A stands for L+U, thus the products of the diagonal is detA
	for (int i = 0; i < 501; ++i)
	{
		DET_A*=MEle[s][i];
	}

	cond = abs(lambda_max/lambda_s);

	//translate to get those close to Miu
	for (int i = 0; i < 39; ++i)
	{
		Miu[i] = lambda1 + (i+1)*(lambda501 - lambda1)/40;
		Recover(A_remain, A);
		//translate
		A->Translate(Miu[i]);

		cout << "lambda_miu" << i+1 << " = "<< setprecision(11) << scientific << InversePowerMethod(A, Epthron)+Miu[i] << endl;
	}

	cout << "lambda501 = " << lambda501 << endl;
	cout << "lambda1 = " << lambda1 << endl;
	cout << "lambda_s = " << lambda_s << endl;
	cout << "cond_A = " << cond << endl;
	cout << "det_A = " <<DET_A << endl;
	
	return 0;
}

int Minimun(int m, int n){
	if (m > n) {
		return n;
	}
	else {
 		return m;
	}
}
int Maximum(int m, int n){
	if (m > n) {
		return m;
	}
	else {
		return n;
	}
}
int Maximum(int o, int p, int q){
	if (o > p)
	{
		if (o > q)
		{
			return o;
		}
		else {
			return q;
		}
	}
	else if (p > q)
	{
		return p;
	}
	else {
		return q;
	}
}
class Matrix{
private:
	int row;
	int column;
	int s, r;
public:
	double **Elements;
	Matrix(int row, int column){
		this->row = row;
		this->column = column;
		Elements = (double**)malloc(sizeof(double*)*row);
		for (int i = 0; i < row; ++i)
		{
			Elements[i] = (double*)malloc(sizeof(double)*column);
		}
	}
	void setValue(){
		this->r = 0;
		this->s = 0;
		double **p = Elements;
		cout << "Input Matrix:\n";
		for (int i = 0; i < this->row; ++i)
		{
			for (int j = 0; j < this->column; ++j)
			{
				cin >> p[i][j];
			}
		}
	}

	//for saving storage
	//s means the number of diagonals that is above the central diagonal
	void setValue(int row, int column, int s, int r){
		double **p = Elements;
		double temp;
		this->s = s;
		this->r = r;
		cout << "Input Matrix:\n";
		for (int i = 0; i < row; ++i)
		{
			for (int j = 0; j < column; ++j)
			{
				cin >> temp;
				if (temp != 0.0)
				{
					p[i-j+s][j] = temp;
				}
			}
		}
	}

	void setR(int r){
		this->r = r;
	}
	void setS(int s){
		this->s = s;
	}
	int getR(){
		return this->r;
	}
	int getS(){
		return this->s;
	}
	int getRow(){
		return this->row;
	}
	int getColumn(){
		return this->column;
	}

	void Translate(double distance){
		if (this->getR() == this->getS() == 0){
			for (int i = 0; i < this->getColumn(); ++i)
			{
				Elements[i][i] -= distance;
			}
		}
		else{
			for (int i = 0; i < this->getColumn(); ++i){
				Elements[this->s][i] -= distance;
			}
		}
	}
};

class Vector{
private:
	int size;
public:
	double* Elements;
	Vector(int size){
		this->size = size;
		Elements = (double*)malloc(sizeof(double)*size);
	}
	void setValue(){
		double *p = Elements;
		cout << "Input Vector:\n";
		for (int i = 0; i < this->size; ++i)
		{
			cin >> p[i];
		}
	}

	int getSize(){
		return this->size;
	}
	double getNorm2(){
		double norm = 0;
		for (int i = 0; i < this->size; i++){
			norm+=Elements[i]*Elements[i];
		}
		return sqrt(norm);
	}
};
class EquationGroup{
private:
	Matrix* A;
	Vector* b;
	Vector* Solution;
public:
	//in order to make storage more economical
	void initEquation(Matrix* A, Vector* b){
		this->A = A;
		this->b = b;
		Solution = new Vector(A->getColumn());
	}

	Vector* getSolution(){
		return Solution;
	}
    
	void TrianglarDecomposition(){
		int n = A->getRow();
		double u, l;
		for(int k = 0; k < n; k++){

			for(int j = k; j < n; j++){
				u = MEle[k][j];
				for(int t = 0; t <= k-1; t++){
					u-=MEle[k][t]*MEle[t][j];
				}

				MEle[k][j] = u;
			}
			for(int i = k+1; i < n; i++){
				l = MEle[i][k];
				for(int t = 0; t <= k-1; t++){
					l-=MEle[i][t]*MEle[t][k];
				}
				l/=MEle[k][k];

				MEle[i][k] = l;
			}
		}
	}
	void Substitution(){
		int n = A->getRow();
		double y[n]; 
		y[0] = VEle[0];
		for(int i = 1; i < n; i++){
			y[i] = VEle[i];
			for(int t = 0; t <= i-1; t++){
				y[i]-=MEle[i][t]*y[t];
			}
		}
		SEle[n-1] = y[n-1]/MEle[n-1][n-1];
		for(int i = n-2; i >= 0; i--){
			SEle[i] = y[i];
			for(int t = i+1; t < n; t++){
				SEle[i]-=MEle[i][t]*SEle[t];
			}
			SEle[i]/=MEle[i][i];
		}
	}
	
	void TrianglarDecompStrip(){
		int s = A->getS();
		int r = A->getR();
		int n = A->getColumn();
		int temp1;
		int temp2;
		for (int k = 0; k < n; k++){
			temp1 = Minimun(k+s, n-1);
			for (int j = k; j <= temp1; j++){
				temp2 = Maximum(0, k-r, j-s);
				for (int t = temp2; t < k; t++){
					MEle[k-j+s][j] -= MEle[k-t+s][t]*MEle[t-j+s][j];
				}
			}
			temp1 = Minimun(k+r, n-1);
			for (int i = k+1; i <= temp1; i++){
				temp2 = Maximum(0, i-r, k-s);
				for (int t = temp2; t < k; t++){
					MEle[i-k+s][k] -= MEle[i-t+s][t]*MEle[t-k+s][k];
				}
				MEle[i-k+s][k] =  MEle[i-k+s][k]/MEle[s][k]; 
			}
		}
	}

	void SubstitutionStrip(){
		int temp;
		int s = A->getS();
		int r = A->getR();
		int n = b->getSize();

		//保护列向量
		Vector* tp = new Vector(n);
		for (int i = 0; i < n; ++i){
			TEle[i] = VEle[i];
		}
		for (int i = 1; i < n; ++i){
			temp = Maximum(0, i-r);
			for (int t = temp; t < i; t++){
				TEle[i] -= MEle[i-t+s][t]*TEle[t]; 
			}
		}
		SEle[n-1] = TEle[n-1]/MEle[s][n-1];    
		for (int i = n-2; i >= 0; i--){
			SEle[i] = TEle[i];
			temp = Minimun(i+s, n-1);
			for (int t = i+1; t <= temp; t++){
				SEle[i] -= MEle[i-t+s][t]*SEle[t];
			}
			SEle[i] = SEle[i]/MEle[s][i];   		
		}
		delete tp;
	}
	
};

double DotProduct(Vector* X, Vector* Y){
	double product = 0;
	for (int i = 0; i < X->getSize(); i++){
		product+=(X->Elements[i]*Y->Elements[i]);
	}
	return product;
}

double PowerMethod(Matrix* A, double Error){
	double Yita;
	double Beta;
	double FormerBeta;
	double LocalError;
	int column = A->getColumn();
	int row = A->getRow();
	//the s of Matrix_A
	int s = A->getS();
	Vector* y = new Vector(column);
	Vector* u = new Vector(column);

	//generate u0
	for (int i = 0; i < column; i++){
		UEle[i] = 0.1;
	}

	//begin loop, until the error is smaller the given one
	do {
		Yita = u->getNorm2();
		for (int i = 0; i < u->getSize(); i++){
			YEle[i] = UEle[i]/Yita;
		}

		//modify u
		if (A->getR() == A->getS() == 0){
			for (int i = 0; i < row; i++){
				UEle[i] = 0;
				for (int j = 0; j < column; j++){
					UEle[i] += MEle[i][j]*YEle[j];
				}
			}
		}
		else {
			for (int i = 0; i < column; i++){
				UEle[i] = 0;
				for (int j = 0; j < column; j++){
					if (i-j+s >= 0 && i-j+s < row)
					{
						UEle[i]+=MEle[i-j+s][j]*YEle[j];
					}
				}
			}

		}

		FormerBeta = Beta;
		Beta = DotProduct(y, u);
		LocalError = abs(Beta - FormerBeta)/abs(Beta);
	} while (LocalError > Error);

	return Beta;
}
double InversePowerMethod(Matrix* A, double Error){
	double Beta;
	double FormerBeta;
	double LocalError;
	double Yita;
	int column = A->getColumn();
	int row = A->getRow();

	Vector* y = new Vector(column);
	Vector* u = new Vector(column);

	EquationGroup* E = new EquationGroup();
    E->initEquation(A, y);

	//do the decomposition
	if (A->getR() == A->getS() == 0){
		E->TrianglarDecomposition();
	}
	else {
		E->TrianglarDecompStrip();
	}

	//generate u0
	for (int i = 0; i < column; i++){
		UEle[i] = 0.1;
	}
	//begin loop, until the error is smaller the given one
	do {
		Yita = u->getNorm2();
		for (int i = 0; i < column; i++){
			YEle[i] = UEle[i]/Yita;
		}

		//solve equation to get u
		E->initEquation(A, y);

		if (A->getR() == A->getS() == 0){
			E->Substitution();
		}
		else {
			E->SubstitutionStrip();
		}

		for (int i = 0; i < column; i++){
			UEle[i] = E->getSolution()->Elements[i];
		}

		FormerBeta = Beta;
		Beta = DotProduct(y, u);
		LocalError = abs(1/Beta - 1/FormerBeta)/abs(1/Beta);
	} while (LocalError > Error);

	return 1/Beta;
}

void Recover(Matrix* A_for_copying, Matrix* A_for_targeting){
	int row = A_for_targeting->getRow();
	int column = A_for_targeting->getColumn();
	for (int i = 0; i < row; ++i)
	{
		for (int j = 0; j < column; ++j)
		{
			A_for_targeting->Elements[i][j] = A_for_copying->Elements[i][j];
		}
	}
}
\end{lstlisting}

\section{打印结果}
\begin{lstlisting}[language={[ANSI]C}]
lambda_miu1 = -1.01829340331e+01
lambda_miu2 = -9.58570742506e+00
lambda_miu3 = -9.17267242393e+00
lambda_miu4 = -8.65228400790e+00
lambda_miu5 = -8.09348380867e+00
lambda_miu6 = -7.65940540769e+00
lambda_miu7 = -7.11968464869e+00
lambda_miu8 = -6.61176433940e+00
lambda_miu9 = -6.06610322659e+00
lambda_miu10 = -5.58510105263e+00
lambda_miu11 = -5.11408352981e+00
lambda_miu12 = -4.57887217687e+00
lambda_miu13 = -4.09647092626e+00
lambda_miu14 = -3.55421121575e+00
lambda_miu15 = -3.04109001813e+00
lambda_miu16 = -2.53397031113e+00
lambda_miu17 = -2.00323076956e+00
lambda_miu18 = -1.50355761123e+00
lambda_miu19 = -9.93558606008e-01
lambda_miu20 = -4.87042673885e-01
lambda_miu21 = 2.23173624957e-02
lambda_miu22 = 5.32417474207e-01
lambda_miu23 = 1.05289896269e+00
lambda_miu24 = 1.58944588188e+00
lambda_miu25 = 2.06033046027e+00
lambda_miu26 = 2.55807559707e+00
lambda_miu27 = 3.08024050931e+00
lambda_miu28 = 3.61362086769e+00
lambda_miu29 = 4.09137851045e+00
lambda_miu30 = 4.60303537828e+00
lambda_miu31 = 5.13292428390e+00
lambda_miu32 = 5.59490634808e+00
lambda_miu33 = 6.08093385703e+00
lambda_miu34 = 6.68035409211e+00
lambda_miu35 = 7.29387744813e+00
lambda_miu36 = 7.71711171424e+00
lambda_miu37 = 8.22522001405e+00
lambda_miu38 = 8.64866606519e+00
lambda_miu39 = 9.25420034458e+00
lambda501 = 9.72463409966e+00
lambda1 = -1.07001136138e+01
lambda_s = -5.55791079436e-03
cond_A = 1.92520427364e+03
det_A = 2.77278614175e+118
\end{lstlisting}
\end{spacing}

\begin{spacing}{1.8}
\section{讨论}
\textbf{对于使用反幂法之后矩阵A改变的考虑}\\
由于使用反幂法时,需要对矩阵A和一个列向量求解线性方程组,而在使用Doolittle分解求解的时候,为了节省存储空间,会把分解后的L和U矩阵存在原本的矩阵A的位置,导致在每次使用反幂法之后,矩阵A都会被改变。本题中使用一个矩阵来保留原矩阵A的值,在每次进行反幂法之后都使用一个函数(源程序492行),把被修改过的A进行恢复。这样既需要多出一个矩阵的存储空间,还需要多次恢复该矩阵,增加了操作次数(尽管和内部求解方程组的时间复杂度$O(n^3)$相比,并不会改变时间复杂度的量级)。而如果在Doolittle分解的时候不把L和U存在A的位置,而是开辟另一块内存空间,多次回带结束之后释放该空间,这样重复使用反幂法的时候也最多只有一个矩阵的空间占用,并且在时间上不需要像上面一样进行一个二重循环,也可以节省算法的操作时间。所以\textbf{对于本题来说},或许可以修改Doolittle算法来避免这个恢复矩阵A的过程。

\end{spacing}
\end{document}